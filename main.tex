\documentclass{book}
\usepackage[utf8]{inputenc}
\usepackage[T1]{fontenc}
\usepackage[paper=a4paper,bottom=2cm,top=2cm,inner=4cm,outer=2cm]{geometry}
\usepackage[french]{babel}
\usepackage[pagestyles]{titlesec}
\usepackage{tcolorbox}
\usepackage{color}
\usepackage{tikz}
\usepackage{pstricks}
\usepackage{amssymb}
\usepackage{amsmath}
\usepackage{mathrsfs}
\usepackage{color}
\usepackage{listings}
\usepackage{calc}
\usepackage{lipsum}
\usepackage{minitoc}
\usepackage{hyperref}

\settitlemarks{part,chapter}

%\newcommand*\circled[1]{\tikz[baseline=(char.base)]{
%		\node[shape=circle,draw,inner sep=2pt](char){#1};
%}
\newcommand{\X}{\phantom{X}}
\newcommand{\circled}[1]{
	\begin{tikzpicture}[baseline, every node/.style={minimum size=8mm, anchor=base}]
	\path node at (0,0) [shape=circle, fill=black!10] (0,0) {\X} 
	node at (1,0) [shape=circle, fill=black!35] (0,0) {\X} 
	node at (2,0) [shape=circle, fill=black!60] (0,0) {#1}
	node at (3,0) [shape=circle, fill=black!35] (0,0) {\X}
	node at (4,0) [shape=circle, fill=black!10] (0,0) {\X}; 
	\end{tikzpicture}
}\newcommand{\circledinv}[1]{%
\begin{tikzpicture}[baseline, every node/.style={minimum size=8mm, anchor=base}]
	\path% 
	node at (0,0) [shape=circle, fill=black!10,minimum size=13mm] (0,0) {\X} 
	node at (0,0) [shape=circle, fill=black!20,minimum size=12mm] (0,0) {\X}
	node at (0,0) [shape=circle, fill=black!40,minimum size=11mm] (0,0) {\X}
	node at (0,0) [shape=circle, fill=black!60,minimum size=10mm] (0,0) {#1} ;
\end{tikzpicture}
}
\DeclareRobustCommand{\bul}\newcommand*\circled[1]{\textcircled{\raisebox{-0.9pt}{#1}}
%}

%\newcommand*\circled[1]{{\fontspec[Ligature=Discretionary]{Junicode}[[#1]]}}

%\newpagestyle{main}{ 
%	\sethead[][\large\bsc{Partie \parttitle}][] %even
%		{}{\large\bsc{\chaptertitle}}{} %odd
%	\setfoot[\bf\textit{\bsc{A bosser}}][\circled{\emph{\thepage}}][A votre service] %even
%		{\emph{Exos-Corigés}}{\circled{\emph{\thepage}}}{\emph{La voie du succes}} %odd
%}
\newpagestyle{main}{ 
	\sethead[][\large\bsc{Partie \parttitle}][] %even
	{}{\large\bsc{\chaptertitle}}{} %odd
	\setfoot[\parttitle~\bul~\chaptertitle][\circled{\bf{\textcolor{white}{\thepage}}}][Even right] %even
	{Odd left}{\circled{\bf{\textcolor{white}{\thepage}}}}{\parttitle~\bul~\chaptertitle} %odd
}
\pagestyle{main}

\headrule 
\setheadrule{0.4pt}
%\setfootrule{0.4pt}

\titleformat{\part}[display] {\normalfont\huge\bfseries}{\centering\Huge\partname}{20pt}{\thispagestyle{empty}\centering\Huge \setcounter{chapter}{0}}

\titleformat{\chapter}[display] {\normalfont\huge\bfseries}{}{20pt}{\thispagestyle{empty}\vspace{8\baselineskip}\centering\Huge\bsc}[\newpage\hbox{}
\vspace*{\fill}
\thispagestyle{empty}
\newpage]

\titleformat{name=\chapter,numberless}[display] {\normalfont\huge\bfseries}{}{20pt}{\thispagestyle{empty}\centering\Huge\bsc}

\renewcommand{\cleardoublepage}{
	\clearpage\ifodd\thepage\else
	\hbox{}
	\vspace*{\fill}
	\thispagestyle{empty}
	\newpage
	\fi
}
\relpenalty=99999
\binoppenalty=99999


%the listings settings
\definecolor{dkgreen}{rgb}{0,0.6,0}
\definecolor{gray}{rgb}{0.5,0.5,0.5}
\definecolor{mauve}{rgb}{0.58,0,0.82}

\lstset{%
	language=C,                  % the language of the code
	basicstyle=\ttfamily\footnotesize,       % the size of the fonts that are used for the code
	numbers=left,                   % where to put the line-numbers
	numberstyle=\tiny\color{gray},  % the style that is used for the line-numbers
	stepnumber=1,                   % the step between two line-numbers. If it's 1, each line 
	% will be numbered
	numbersep=5pt,                  % how far the line-numbers are from the code
	backgroundcolor=\color{white},  % choose the background color. You must add \usepackage{color}
	showspaces=false,               % show spaces adding particular underscores
	showstringspaces=false,         % underline spaces within strings
	showtabs=false,                 % show tabs within strings adding particular underscores
	frame=single,                   % adds a frame around the code
	rulecolor=\color{black},        % if not set, the frame-color may be changed on line-breaks within not-black text (e.g. commens (green here))
	tabsize=4,                      % sets default tabsize to 2 spaces
	captionpos=b,                   % sets the caption-position to bottom
	breaklines=true,                % sets automatic line breaking
	breakatwhitespace=false,        % sets if automatic breaks should only happen at whitespace
	%title=\lstname,                 % show the filename of files included with \lstinputlisting;
	%also try caption instead of title
	keywordstyle=\color{blue},          % keyword style
	commentstyle=\color{dkgreen},       % comment style
	stringstyle=\color{green},         % string literal style
	escapeinside={\%*}{*)},            % if you want to add a comment within your code
	morekeywords={*,...}               % if you want to add more keywords to the set
}
%the tcolorbox settings
\tcbuselibrary{skins,xparse,breakable}
\usetikzlibrary{patterns}

\newcounter{counterexo}
\numberwithin{counterexo}{section}

\newcommand*{\addcor}{
}
%\immediate\newwrite\exocor
%\immediate\openout\exocor=corrections.tex

\NewTotalTColorBox{\solution}{mmm}{%
	breakable,
	enhanced,
	fonttitle=\bfseries,
	width=\linewidth-6pt, 
	enlarge top by=3pt,
	enlarge bottom by=3pt, 
	enlarge left by=3pt,
	enlarge right by=3pt,
	boxrule=0pt,top=1mm,bottom=1mm,
	colframe=black!50,
	colbacktitle=black!5,
	coltitle=black,
	colback=white, 
	borderline={0.5pt}{-0.5pt}{black!30}, 
	borderline={1pt}{-3pt}{black!75},
	%	varwidth boxed title
	title={Exercice~\ref{#1:exercise@#2} page~\pageref{#1:exercise@#2}}, 
	phantomlabel={#1:solution@#2}, 
	%attach title to upper=\par,
}{\input{#3}}

\newenvironment{Exolist}[1]
{
	\newcommand{\cor}{\tcblower}
	\NewTColorBox[use counter=counterexo]{exercise}{+O{}}{%
		enhanced,
		enhanced jigsaw,opacityback=0.35,
		breakable,
		before skip=2mm,
		after skip=2mm, 
		colback=white,
		colbacktitle=black!15,
		colframe=black!50,
		boxrule=0.2mm,
		%overlay unbroken and first={\path[left color=tcbcol@back!60!black,right color=tcbcol@back!60!black, middle color=tcbcol@back!80!black] ([xshift=-0.2mm,yshift=-1.02cm]frame.north east) -- ++(-1,1) -- ++(-0.5,0) -- ++(1.5,-1.5) -- cycle;},
		attach boxed title to top left={xshift=1cm,yshift*=1mm-\tcboxedtitleheight}, 
		%varwidth boxed title*=-3cm, 
		boxed title style={
			frame code={ \path[fill=tcbcol@back!90] ([yshift=-1mm,xshift=-1mm]frame.north west) arc[start angle=0,end angle=180,radius=1mm] ([yshift=-1mm,xshift=1mm]frame.north east) arc[start angle=180,end angle=0,radius=1mm]; \path[left color=tcbcol@back!60!black,right color=tcbcol@back!60!black, middle color=tcbcol@back!80!black] ([xshift=-2mm]frame.north west) -- ([xshift=2mm]frame.north east) [rounded corners=1mm]-- ([xshift=1mm,yshift=-1mm]frame.north east) -- (frame.south east) -- (frame.south west) -- ([xshift=-1mm,yshift=-1mm]frame.north west) [sharp corners]-- cycle; },
			interior engine=empty, }, 
		fonttitle=\bfseries,
		title={Exercice~\arabic{\tcbcounter}}, 
		label={#1:exercise@\thetcbcounter},
		%	attach title to upper=\quad, 
		after upper={\par\hfill\textcolor{black!90}%
			{\itshape Solution page~\pageref{#1:solution@\thetcbcounter}}
		},
		lowerbox=ignored,
		savelowerto=#1/solutions/exercise-\thetcbcounter.tex, record={\string\solution{#1}{\thetcbcounter}{#1/solutions/exercise-\thetcbcounter.tex}},
		##1
	}
	
	\tcbset{nosol/.style={no recording,after upper=,overlay unbroken and first=}}
	\tcbset{from/.style={before upper={\par\textcolor{black!90}  {\textbf{\itshape Source: ##1}}\par}}}
	\tcbrecord{\string\chapter{\chaptertitle}}
	}
{
}

%my amsmath command
\newcommand{\mbb}[1]
{\ensuremath{\mathbb{#1}}}
\newcommand{\mcal}[1]
{\ensuremath{\mathcal{#1}}}

\tcbstartrecording[corrections.tex]
\begin{document}
	\doparttoc
	\tableofcontents
	\part{Enoncés}
	\parttoc
	
	\chapter{Algebre}
	\begin{Exolist}{algebre}	
		\begin{exercise}[from=Devoir MPI 2019,nosol]
	Les questions de cet exercice sont indépendantes.
	\begin{enumerate}
		\item Soient $A,B,C$ des parties d'un ensemble $E$. Montrer que
		$$
			(A \Delta B) \cap C=(A \cap C) \Delta(B \cup C).
		$$
		\item Déterminer $f^{-1}([-3,3])$ pour la fonction $f:\mathbb{R}\rightarrow\mathbb{R},\ x \mapsto x^2$.
		\item Soient $E,F$ des ensembles, $f:E \rightarrow F$ une application et $A \subset E,\ B \subset F$.
		Montrer que $f(A \cap f^{-1}(B))=f(A) \cap B$.
		\item Soit $E$ un ensemble. On définit sur $\mathcal{P}(E)$ la relation $\mathcal{R}$ par
		$$
			A \mathcal{R} B \Longleftrightarrow A = B \text{ ou } A = \bar{B}.
		$$
		Montrer que $\mathcal{R}$ est une relation d'équivalence sur $\mathcal{P}(E)$.
	\end{enumerate}
\end{exercise}

\begin{exercise}[from=Devoir MPI 2019,nosol]
	Soit $G$ un groupe et $H,K$ deux sous-groupe de $G$. On note
	$$
		HK=hk \in G \mid h \in H,k \in K.
	$$
	\begin{enumerate}
		\item Montrer que $HK$ est un sous groupe de $G$ si et seulement si $HK=KH$.
		\item Quel est le cardinal de $HK$ lorsque $H$ et $K$ sont finis ?
		\item Montrer que si $H \cap K = \{e\}$ alors tout élément de $HK$ s'écrit de manière unique comme produit $hk$.
		\item Soit $G=\mathbb{Z}/6\mathbb{Z}, H=\langle \bar{2}\rangle \text{ et } K=\langle\bar{3}\rangle$. Vérifier que $G=HK$ et qu'il y a unicité de l'écriture comme dans la question précédente.
	\end{enumerate}
\end{exercise}

\begin{exercise}[from=Devoir MPI 2017,nosol]
	Soit $\mathcal{F}$ une famille non vide de parties d'un ensemble $E$ telle que:
	\[
		\forall (A,B) \in (\mathcal{P}(E))^2,\exists~ C \in \mathcal{F} \text{ tel que } C \subset A \cap B.
	 \]
	 Dans $\mathcal{P}(E)$, on définit la relation $\mathcal{R}$ par:
	 \[
	 	X\mathcal{R}Y\Leftrightarrow \exists A \in \mathcal{F} \text{ tel que } A \cap X = A \cap Y.
	 \]
	 \begin{enumerate}
	 	\item Montrer que $\mathcal{R}$ est une relation d'équivalence dans $\mathcal{P}(E)$.
	 	\item  Montrer que pour tout $X \in \mathcal{P}(E)$, on a
	 	\[
	 		cl(X)=\{Y \in \mathcal{P}(E)/Y=(A \cap X) \cup B , \text{ avec } A \in \mathcal{F} \text{ et } B \subset A^C \}.
	 	\]
	 \end{enumerate}
\end{exercise}
\begin{exercise}[from=Devoir MPI 2017]
	On définit la loi $\star$ sur $\mathbb{R}$ en posant: $x \star y = x+y-xy$.
	\begin{enumerate}
		\item Étudier la loi $\star$. $(\mbb{R}, \star)$ est-il un groupe?
		\item Montrer que $(\mbb{R}-\{1\},\star)$ est un groupe abélien isomorphe à $(\mbb{R}^*,\times)$. (On pourra considérer l'application $\varphi: \mbb{R}-\{1\}\rightarrow \mbb{R}^*,t \mapsto 1-t$)
	\end{enumerate}
	\cor
	\begin{enumerate}
		\item La loi est commutative , associative d'élement neutre 0.
		\begin{align*}
			xx'& = x+x'-xx'= 0\\
			   & \Leftrightarrow x'(1-x)=-x
		\end{align*}
		$x=1$ n'as pas d'inverse. $(\mathbb{R},\star)$ n'est pas un groupe.
		\item $\forall x,y \in \mathbb{R}$, $1-x \star y = (1-x)(1-y)$.
		\\On a $x=1-x \text{ et } y=1-y$.
		\\Soit $\varphi(xy)=\varphi(x) \star \varphi(y)$
	\end{enumerate}
\end{exercise}

\begin{exercise}[from=Devoir SR MPI 2010]
	Soient $E$ un ensemble, $A$ et $B$ deux parties de $E$. On définit l'application:
	\begin{align*}
		\Phi:  \mathcal{P}(E)&\longrightarrow \mathcal{P}(A)\times \mathcal{P}(B)\\
		X &\longmapsto (X \cap A, X \cap B)
	\end{align*}
	\begin{enumerate}
		\item Montrer que pour que $\Phi$ soit injective, \textit{il faut et il suffit} que $A\cup B = E$
		\item Montrer que pour que $\Phi$ soit surjective, \textit{il faut et il suffit} que $A\cap B = \varnothing$
		\item Lorque $\Phi$ est bijective, determiner son application reciproque.
	\end{enumerate}

	\cor
	\begin{enumerate}
		\item On suppose que $\Phi$ soit injective. En remarquant que $\Phi(A \cup B) =(A,B)=\Phi(E)$. Alors $(X\cap A, X \cap B) = (Y \cap A, Y \cap B).$ Donc $X \cap A = Y \cap A$ et $X \cap B = Y \cap B$. Par suite $(X \cap A) \cup (X \cap B) = (Y \cap A) \cup (Y \cap B)$ ou encore $X\cap (A \cup B) = Y \cap(A \cup B)$. Enfin, puisque $E=A \cup B$, il  vient que $X=Y$. Donc $\Phi$ est injective.
		\item Supposons que $\Phi$ est surjective. Alors il existe $X \in \mathcal{P}(E)$ tel que $\Phi(X)=(\varnothing, B)$. D'où $(X \cap A, X \cap B)=(\varnothing, B)$. Donc $X \cap A = \varnothing$ et $X\cap B = B$. Ainsi, $B \subset X$ et $X \cap A = \varnothing$. Par suite $A\cap B = \varnothing$. Réciproquement, supposons que $A\cap B=\varnothing$ et soit $(A_1, B_1)\in \mathcal{P}(A)\times \mathcal{P}(B)$. On cherche $X\in \mathcal{P}(E)$ tel que $\Phi (X)=(A_1,B_1)$. Donc $X\cap A =A_1$ et $X\cap B = B_1$. Ainsi $A_1 \subset X$ et $B_1 \subset X$. Si l'on choisit $X$ tel que $A_1 \subset X$  et $B_1 \subset X$ alors $\Phi (X)=(A_1,B_1)$. Par exemple $\Phi (A_1\cup B_1)=(A_1,B_1)$ car $A\cap B=\varnothing$.
		\item Soit $(A_1, B_1)\in \mathcal{P}(A)\times \mathcal{P}(B)$. Comme $\Phi$ est surjective, il existe $X\in \mathcal{P}(E)$ tel que $\Phi (X)=(A_1,B_1)$. Donc $X\cap A =A_1$ et $X\cap B = B_1$. Donc $A_1 \cup B_1 =(X\cap A)\cup (X\cap B) = X\cap (A\cup B)$. Comme $\Phi$ est injective, $A\cup B=E$. Donc $X=A_1\cup B_1$. Par suite, la réciproque de $\Phi$ est définie par:
		\begin{align*}
			\Phi^{-1}: \mathcal{P}(A)\times \mathcal{P}(B) &\longrightarrow \mathcal{P}(E)\\
			(X,Y) &\longmapsto X\cup Y
		\end{align*}
	\end{enumerate}
\end{exercise}

\begin{exercise}

\end{exercise}

\begin{exercise}[from=Devoir SR MPI 2016]
	%	\pscirclebox{stuff}
	\lipsum[2-5]
	\tcblower
	\lipsum[3]
\end{exercise}

\begin{exercise}[nosol,from=Devoir SR MPI 2010]
	\lipsum[6]
\end{exercise}
\begin{exercise}
	\lipsum[2-5]
	\tcblower
	\lipsum[3]
\end{exercise}
\begin{exercise}[from=Devoir SN MPI 2018]
	\lipsum[2-5]
	\tcblower
	\lipsum[3]
\end{exercise}
\begin{exercise}[nosol,from=Concours Asecna 2015]
	\lipsum[6]
\end{exercise}
\begin{exercise}
	\lipsum[2-5]
	\tcblower
	\lipsum[3-10]
\end{exercise}

\begin{exercise}[nosol]
	\lipsum[6]
\end{exercise}
\begin{exercise}
	\lipsum[2-5]
	\tcblower
	\lipsum[3]
\end{exercise}
\begin{exercise}[from=TD SB 2015]
	\lipsum[2-5]
	\tcblower
	\lipsum[3]
\end{exercise}
\begin{exercise}[nosol]
	\lipsum[6]
\end{exercise}
\begin{exercise}
	\lipsum[2-5]
	\tcblower
	\lipsum[3]
\end{exercise}

\begin{exercise}[from=TD MPI 2015,nosol]
	\lipsum[6]
\end{exercise}
\begin{exercise}
	\lipsum[2-5]
	\tcblower
	\lipsum[3]
\end{exercise}
\begin{exercise}[from=Devoir SR MPI 2019]
	\lipsum[2-5]
	\tcblower
	\lipsum[3]
\end{exercise}
\begin{exercise}[nosol]
	\lipsum[6]
\end{exercise}
\begin{exercise}
	\lipsum[2-5]
	\tcblower
	\lipsum[3]
\end{exercise}

\begin{exercise}[nosol,from=Devoir SN MPI 2012]
	\lipsum[6]
\end{exercise}
\begin{exercise}
	\lipsum[2-5]
	\tcblower
	\lipsum[3]
\end{exercise}
\begin{exercise}[from=Devoir SN SEG 2018]
	\lipsum[2-5]
	\tcblower
	\lipsum[3]
\end{exercise}
\begin{exercise}[nosol]
	\lipsum[6]
\end{exercise}

	\end{Exolist}

	\chapter{Analyse}
	\begin{Exolist}{analyse}	
		\input{analyse.tex}
	\end{Exolist}

	\chapter{Informatique}
	\begin{Exolist}{informatique}
	\input{informatique.tex}
	\end{Exolist}

	\chapter{Physique}
	\begin{Exolist}{physique}	
		\begin{exercise}[from=Devoir SR MPI 2016]
	%	\pscirclebox{stuff}
	\lipsum[2-5]
	\tcblower 
	\lipsum[3]
\end{exercise}

\begin{exercise}[from=Devoir SR MPI 2016]
	%	\pscirclebox{stuff}
	\lipsum[2-5]
	\tcblower 
	\lipsum[3]
\end{exercise}

\begin{exercise}[nosol,from=Devoir SR MPI 2010]
	\lipsum[6]
\end{exercise}
\begin{exercise} 
	\lipsum[2-5]
	\tcblower 
	\lipsum[3]
\end{exercise}
\begin{exercise}[from=Devoir SN MPI 2018]
	\lipsum[2-5]
	\tcblower 
	\lipsum[3]
\end{exercise}
\begin{exercise}[nosol,from=Concours Asecna 2015]
	\lipsum[6]
\end{exercise}
\begin{exercise} 
	\lipsum[2-5]
	\tcblower 
	\lipsum[3-10]
\end{exercise}

\begin{exercise}[nosol]
	\lipsum[6]
\end{exercise}
\begin{exercise} 
	\lipsum[2-5]
	\tcblower 
	\lipsum[3]
\end{exercise}
\begin{exercise}[from=TD SB 2015]
	\lipsum[2-5]
	\tcblower 
	\lipsum[3]
\end{exercise}
\begin{exercise}[nosol]
	\lipsum[6]
\end{exercise}
\begin{exercise} 
	\lipsum[2-5]
	\tcblower 
	\lipsum[3]
\end{exercise}

\begin{exercise}[from=TD MPI 2015,nosol]
	\lipsum[6]
\end{exercise}
\begin{exercise} 
	\lipsum[2-5]
	\tcblower 
	\lipsum[3]
\end{exercise}
\begin{exercise}[from=Devoir SR MPI 2019]
	\lipsum[2-5]
	\tcblower 
	\lipsum[3]
\end{exercise}
\begin{exercise}[nosol]
	\lipsum[6]
\end{exercise}
\begin{exercise} 
	\lipsum[2-5]
	\tcblower 
	\lipsum[3]
\end{exercise}

\begin{exercise}[nosol,from=Devoir SN MPI 2012]
	\lipsum[6]
\end{exercise}
\begin{exercise} 
	\lipsum[2-5]
	\tcblower 
	\lipsum[3]
\end{exercise}
\begin{exercise}[from=Devoir SN SEG 2018]
	\lipsum[2-5]
	\tcblower 
	\lipsum[3]
\end{exercise}
\begin{exercise}[nosol]
	\lipsum[6]
\end{exercise}
 	
	\end{Exolist}

	\tcbstoprecording	
	\part{Corrigés}
	\parttoc
	\chapter{Algebre}
\solution{algebre}{4}
\solution{algebre}{5}
\solution{algebre}{6}
\solution{algebre}{7}
\solution{algebre}{8}
\solution{algebre}{9}
\solution{algebre}{10}
\solution{algebre}{12}
\solution{algebre}{14}
\solution{algebre}{15}
\solution{algebre}{17}
\solution{algebre}{19}
\solution{algebre}{20}
\solution{algebre}{22}
\solution{algebre}{24}
\solution{algebre}{25}
\chapter{Analyse}
\solution{analyse}{1}
\solution{analyse}{2}
\solution{analyse}{4}
\solution{analyse}{5}
\solution{analyse}{7}
\solution{analyse}{9}
\solution{analyse}{10}
\solution{analyse}{12}
\solution{analyse}{14}
\solution{analyse}{15}
\solution{analyse}{17}
\solution{analyse}{19}
\solution{analyse}{20}
\chapter{Informatique}
\solution{informatique}{1}
\solution{informatique}{2}
\solution{informatique}{4}
\solution{informatique}{5}
\solution{informatique}{7}
\solution{informatique}{9}
\solution{informatique}{10}
\solution{informatique}{12}
\solution{informatique}{14}
\solution{informatique}{15}
\solution{informatique}{17}
\solution{informatique}{19}
\solution{informatique}{20}
\chapter{Physique}
\solution{physique}{1}
\solution{physique}{2}
\solution{physique}{4}
\solution{physique}{5}
\solution{physique}{7}
\solution{physique}{9}
\solution{physique}{10}
\solution{physique}{12}
\solution{physique}{14}
\solution{physique}{15}
\solution{physique}{17}
\solution{physique}{19}
\solution{physique}{20}

\end{document}