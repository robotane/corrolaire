\begin{exercise}[from=Devoir MPI 2019,nosol]
	Les questions de cet exercice sont indépendantes.
	\begin{enumerate}
		\item Soient $A,B,C$ des parties d'un ensemble $E$. Montrer que
		$$
			(A \Delta B) \cap C=(A \cap C) \Delta(B \cup C).
		$$
		\item Déterminer $f^{-1}([-3,3])$ pour la fonction $f:\mathbb{R}\rightarrow\mathbb{R},\ x \mapsto x^2$.
		\item Soient $E,F$ des ensembles, $f:E \rightarrow F$ une application et $A \subset E,\ B \subset F$.
		Montrer que $f(A \cap f^{-1}(B))=f(A) \cap B$.
		\item Soit $E$ un ensemble. On définit sur $\mathcal{P}(E)$ la relation $\mathcal{R}$ par
		$$
			A \mathcal{R} B \Longleftrightarrow A = B \text{ ou } A = \overline{B}.
		$$
		Montrer que $\mathcal{R}$ est une relation d'équivalence sur $\mathcal{P}(E)$.
	\end{enumerate}
\end{exercise}

\begin{exercise}[from=Devoir MPI 2019,nosol]
	Soit $G$ un groupe et $H,K$ deux sous-groupes de $G$. On note
	$$
		HK=hk \in G \mid h \in H,k \in K.
	$$
	\begin{enumerate}
		\item Montrer que $HK$ est un sous groupe de $G$ si et seulement si $HK=KH$.
		\item Quel est le cardinal de $HK$ lorsque $H$ et $K$ sont finis ?
		\item Montrer que si $H \cap K = \{e\}$ alors tout élément de $HK$ s'écrit de manière unique comme produit $hk$.
		\item Soit $G=\mathbb{Z}/6\mathbb{Z}, H=\langle \bar{2}\rangle \text{ et } K=\langle\bar{3}\rangle$. Vérifier que $G=HK$ et qu'il y a unicité de l'écriture comme dans la question précédente.
	\end{enumerate}
\end{exercise}

\begin{exercise}[from=Devoir MPI 2017,nosol]
	Soit $\mathcal{F}$ une famille non vide de parties d'un ensemble $E$ telle que:
	\[
		\forall (A,B) \in (\mathcal{P}(E))^2,\exists~ C \in \mathcal{F} \text{ tel que } C \subset A \cap B.
    \]
    Dans $\mathcal{P}(E)$, on définit la relation $\mathcal{R}$ par:
    \[
    X\mathcal{R}Y\Leftrightarrow \exists A \in \mathcal{F} \text{ tel que } A \cap X = A \cap Y.
    \]
    \begin{enumerate}
    \item Montrer que $\mathcal{R}$ est une relation d'équivalence dans $\mathcal{P}(E)$.
    \item  Montrer que pour tout $X \in \mathcal{P}(E)$, on a
    \[
        cl(X)=\{Y \in \mathcal{P}(E)/Y=(A \cap X) \cup B , \text{ avec } A \in \mathcal{F} \text{ et } B \subset A^C \}.
    \]
    \end{enumerate}
\end{exercise}
\begin{exercise}[from=Devoir MPI 2017]
	On définit la loi $\star$ sur $\mathbb{R}$ en posant: $x \star y = x+y-xy$.
	\begin{enumerate}
		\item Étudier la loi $\star$. $(\mbb{R}, \star)$ est-il un groupe?
		\item Montrer que $(\mbb{R}-\{1\},\star)$ est un groupe abélien isomorphe à $(\mbb{R}^*,\times)$. (On pourra considérer l'application $\varphi: \mbb{R}-\{1\}\rightarrow \mbb{R}^*,t \mapsto 1-t$)
	\end{enumerate}
	\cor
	\begin{enumerate}
		\item La loi est commutative, associative d'élément neutre 0.
		\begin{align*}
			xx'& = x+x'-xx'= 0\\
			   & \Leftrightarrow x'(1-x)=-x
		\end{align*}
		$x=1$ n'as pas d'inverse. $(\mathbb{R},\star)$ n'est pas un groupe.
		\item $\forall x,y \in \mathbb{R}$, $1-x \star y = (1-x)(1-y)$.
		\\On a $x=1-x \text{ et } y=1-y$.
		\\Soit $\varphi(xy)=\varphi(x) \star \varphi(y)$
	\end{enumerate}
\end{exercise}

\begin{exercise}[from=Devoir SR MPI 2010]
	Soient $E$ un ensemble, $A$ et $B$ deux parties de $E$. On définit l'application :
	\begin{align*}
		\Phi:  \mathcal{P}(E)&\longrightarrow \mathcal{P}(A)\times \mathcal{P}(B)\\
		X &\longmapsto (X \cap A, X \cap B)
	\end{align*}
	\begin{enumerate}
		\item Montrer que pour que $\Phi$ soit injective, \textit{il faut et il suffit} que $A\cup B = E$
		\item Montrer que pour que $\Phi$ soit surjective, \textit{il faut et il suffit} que $A\cap B = \varnothing$
		\item Lorsque $\Phi$ est bijective, déterminer son application réciproque.
	\end{enumerate}

	\cor
	\begin{enumerate}
		\item On suppose que $\Phi$ soit injective. En remarquant que $\Phi(A \cup B) =(A,B)=\Phi(E)$. Alors $(X\cap A, X \cap B) = (Y \cap A, Y \cap B).$ Donc $X \cap A = Y \cap A$ et $X \cap B = Y \cap B$. Par suite $(X \cap A) \cup (X \cap B) = (Y \cap A) \cup (Y \cap B)$ ou encore $X\cap (A \cup B) = Y \cap(A \cup B)$. Enfin, puisque $E=A \cup B$, il  vient que $X=Y$. Donc $\Phi$ est injective.
		\item Supposons que $\Phi$ est surjective. Alors il existe $X \in \mathcal{P}(E)$ tel que $\Phi(X)=(\varnothing, B)$. D'où $(X \cap A, X \cap B)=(\varnothing, B)$. Donc $X \cap A = \varnothing$ et $X\cap B = B$. Ainsi, $B \subset X$ et $X \cap A = \varnothing$. Par suite $A\cap B = \varnothing$. Réciproquement, supposons que $A\cap B=\varnothing$ et soit $(A_1, B_1)\in \mathcal{P}(A)\times \mathcal{P}(B)$. On cherche $X\in \mathcal{P}(E)$ tel que $\Phi (X)=(A_1,B_1)$. Donc $X\cap A =A_1$ et $X\cap B = B_1$. Ainsi $A_1 \subset X$ et $B_1 \subset X$. Si l'on choisit $X$ tel que $A_1 \subset X$  et $B_1 \subset X$ alors $\Phi (X)=(A_1,B_1)$. Par exemple $\Phi (A_1\cup B_1)=(A_1,B_1)$, car $A\cap B=\varnothing$.
		\item Soit $(A_1, B_1)\in \mathcal{P}(A)\times \mathcal{P}(B)$. Comme $\Phi$ est surjective, il existe $X\in \mathcal{P}(E)$ tel que $\Phi (X)=(A_1,B_1)$. Donc $X\cap A =A_1$ et $X\cap B = B_1$. Donc $A_1 \cup B_1 =(X\cap A)\cup (X\cap B) = X\cap (A\cup B)$. Comme $\Phi$ est injective, $A\cup B=E$. Donc $X=A_1\cup B_1$. Par suite, la réciproque de $\Phi$ est définie par :
		\begin{align*}
			\Phi^{-1}: \mathcal{P}(A)\times \mathcal{P}(B) &\longrightarrow \mathcal{P}(E)\\
			(X,Y) &\longmapsto X\cup Y
		\end{align*}
	\end{enumerate}
\end{exercise}

\begin{exercise}[from=Devoir MPI 2010]
Soit $E$ un ensemble.
\begin{enumerate}
    \item Montrer que la relation binaire $\mathcal{R}$ définie par
    \begin{equation*}
        \forall X, Y \in \mathcal{P}(E), X \mathcal{R}Y \Longleftrightarrow X \in \{Y, \overline{Y}\}
    \end{equation*}
    est une relation d'équivalence. On rappelle que $\overline{Y}$ désigne le complémentaire de $Y$ dans $E$.
    \item Déterminer la classe d'un élément $X$ de $\mathcal{P}(E)$
    \item On suppose que $E$ est fini. Déterminer le cardinal de l'ensemble quotient $\mathcal{P}(E)/\mathcal{R}$.
    \item Déterminer $\mathcal{P}(E)/\mathcal{R}$ pour $E=\{1,2,3\}$.
\end{enumerate}

\cor

\begin{enumerate}
    \item \begin{enumerate}
        \item On a $X \in \{X, \overline{X}\}$ pour tout $X \in \mathcal{P}(E)$. Donc $\forall X \in \mathcal{P}(E), X\mathcal{R}Y$. La relation est donc réflexive.
        \item Soient $X,Y \in \mathcal{P}(E)$ tels que $X\mathcal{R}Y$. Alors $X \in \{Y, \overline{Y}\}$. Donc $Y \in \{Y, \overline{Y}\}$. Ainsi, $Y\mathcal{R}X$ et $\mathcal{R}$ est symétrique.
        \item Soient $X,Y \text{ et } Z \in \mathcal{P}(E)$ tels que $X\mathcal{R}Y \text{ et } Y\mathcal{R} Z$. Donc $X \in \{Y, \overline{Y}\}$ et $Y \in \{Z, \overline{Z}\}$. Remarquons que $Y \in \{Z, \overline{Z}\}$ si et seulement si $\{Y, \overline{Y}\} =\{Z, \overline{Z}\}$. Par suite $X \in \{Z, \overline{Z}\}$ et par conséquent $X\mathcal{R}Z$. Donc la relation $\mathcal{R}$ est transitive.
    \end{enumerate}
    \item $\dot{X}=\{X, \overline{X}\}$
    \item Observons que $E$ est fini non vide, pour tout $X \in \mathcal{P}(E)$, on a $X \neq \overline{X}$. Donc $Card(\dot{X})=2$. Comme $\mathcal{P}(E)$ est réunion de classes d'équivalence, on a:
    \begin{equation*}
        Card(\mathcal{P}(E)/\mathcal{R})=\frac{Card(\mathcal{P}(E))}{2}= 2^{n-1} \text{ avec } Card(E)=n.
    \end{equation*}
    \item \begin{equation*}
        \mathcal{P}(E)/\mathcal{R} = \Big\{\big\{\varnothing, E\big\}; \big\{\{1\}, \{2, 3\}\big\}; \big\{\{3\}, \{1, 2\}\big\};\Big\}
    \end{equation*}
\end{enumerate}

\end{exercise}

\begin{exercise}[from=Devoir SR MPI 2010]
    \begin{enumerate}
        \item Montrer que la relation binaire $S$ définie sur $\mathbb{N}^*$ par
        \begin{equation*}
            \forall x,y \in \mathbb{N}^*, xSy \Longleftrightarrow x \text{ divise } y
        \end{equation*}
        est une relation d'ordre.
        \item La relation $S$ est-elle une relation d'ordre total ou partiel?
        \item Déterminer l'ensemble des minorants et l'ensemble  des majorants de chacun des sous-ensembles suivants de $\mathbb{N}^*$ pour la relation $S$:
        \begin{equation*}
            A=\{1,2,3,4,5,6\}; \ B=\{2,4,8,16,62,64,128\}; \ C=\{16,32,40,48\}.
        \end{equation*}
        \item Si $A$ est un sous-ensemble non vide de $\mathbb{N}^*$, déterminer la borne supérieure et la borne inférieure de $A$.
    \end{enumerate}

    \cor
    \begin{enumerate}
        \item On montre sans difficulté que $S$ est réflexive, antisymétrique et transitive.
        \item La relation $S$ est une relation d'ordre partiel car, par exemple, 2 et 3 ne sont pas comparables par $S$.
        \item Ensemble: \begin{itemize}
            \item des majorants de $A:\ \{60, 120, \cdots, 60k, \cdots\}$, ce sont les multiples non nuls de 60,
            \item des majorants de $B:\ \{128, 256, \cdots, 128k, \cdots\}$, ce sont les multiples non nuls de 128,
            \item des majorants de $C:\ \{480, 960, \cdots, 480k, \cdots\}$, ce sont les multiples non nuls de 480;
        \end{itemize}
        \item Ensemble: \begin{itemize}
            \item des minorants de $A:\ \{1\}$,
            \item des minorants de $B:\ \{1, 2\}$,
            \item des minorants de $C:\ \{1, 2, 4, 8, 16\}$.
        \end{itemize}
        \item \begin{itemize}
            \item  $\text{bornesup}(A)=ppcm$ des éléments de $A$;
            \item  $\text{borneinf}(A)=pgcd$ des éléments de $A$;
        \end{itemize}
    \end{enumerate}
\end{exercise}

\begin{exercise}[from=Devoir SR MPI 2010]
	On pose $G=\{(x,y)\in \mathbb{R}^2 \vert  x\neq 1 \text{ et } y \neq -1\}$.

    On définit une loi $*$ par : $\forall (x_1, y_1) \in G,\ \forall (x_2,y_2) \in G$,
    \begin{equation*}
        (x_1, y_1)*(x_2,y_2)=(x_1+x_2-x_1x_2, y_1+y_2+y_1y_2)
    \end{equation*}
    \begin{enumerate}
        \item Vérifier que la loi $*$ est interne dans $G$ et montrer que $G$ muni de cette loi est un groupe commutatif.
        \item Calculer, pour tout entier naturel non nul $n$ et pour tout $(x_1,x_2) \in G$,
        \begin{equation*}
            (x_1,x_2)^{[n]}=\underbrace{(x_1,x_2)*(x_1,x_2)*\cdots * (x_1,x_2)}_{n \text{ fois}}
        \end{equation*}
    \end{enumerate}

    \cor
    \begin{enumerate}
        \item L'ensemble
    \end{enumerate}

    Soit $x \text{ et } y$ deux nombres.
\end{exercise}

\begin{exercise}
	\lipsum[2-5]
	\tcblower
	\lipsum[3]
\end{exercise}
\begin{exercise}[from=Devoir SN MPI 2018]
	\lipsum[2-5]
	\tcblower
	\lipsum[3]
\end{exercise}
\begin{exercise}[nosol,from=Concours Asecna 2015]
	\lipsum[6]
\end{exercise}
\begin{exercise}
	\lipsum[2-5]
	\tcblower
	\lipsum[3-10]
\end{exercise}

\begin{exercise}[nosol]
	\lipsum[6]
\end{exercise}
\begin{exercise}
	\lipsum[2-5]
	\tcblower
	\lipsum[3]
\end{exercise}
\begin{exercise}[from=TD SB 2015]
	\lipsum[2-5]
	\tcblower
	\lipsum[3]
\end{exercise}
\begin{exercise}[nosol]
	\lipsum[6]
\end{exercise}
\begin{exercise}
	\lipsum[2-5]
	\tcblower
	\lipsum[3]
\end{exercise}

\begin{exercise}[from=TD MPI 2015,nosol]
	\lipsum[6]
\end{exercise}
\begin{exercise}
	\lipsum[2-5]
	\tcblower
	\lipsum[3]
\end{exercise}
\begin{exercise}[from=Devoir SR MPI 2019]
	\lipsum[2-5]
	\tcblower
	\lipsum[3]
\end{exercise}
\begin{exercise}[nosol]
	\lipsum[6]
\end{exercise}
\begin{exercise}
	\lipsum[2-5]
	\tcblower
	\lipsum[3]
\end{exercise}

\begin{exercise}[nosol,from=Devoir SN MPI 2012]
	\lipsum[6]
\end{exercise}
\begin{exercise}
	\lipsum[2-5]
	\tcblower
	\lipsum[3]
\end{exercise}
\begin{exercise}[from=Devoir SN SEG 2018]
	\lipsum[2-5]
	\tcblower
	\lipsum[3]
\end{exercise}
\begin{exercise}[nosol]
	\lipsum[6]
\end{exercise}
