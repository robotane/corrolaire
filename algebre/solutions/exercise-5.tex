	\begin{enumerate}
		\item On suppose que $\Phi$ soit injective. En remarquant que $\Phi(A \cup B) =(A,B)=\Phi(E)$. Alors $(X\cap A, X \cap B) = (Y \cap A, Y \cap B).$ Donc $X \cap A = Y \cap A$ et $X \cap B = Y \cap B$. Par suite $(X \cap A) \cup (X \cap B) = (Y \cap A) \cup (Y \cap B)$ ou encore $X\cap (A \cup B) = Y \cap(A \cup B)$. Enfin, puisque $E=A \cup B$, il  vient que $X=Y$. Donc $\Phi$ est injective.
		\item Supposons que $\Phi$ est surjective. Alors il existe $X \in \mathcal{P}(E)$ tel que $\Phi(X)=(\varnothing, B)$. D'où $(X \cap A, X \cap B)=(\varnothing, B)$. Donc $X \cap A = \varnothing$ et $X\cap B = B$. Ainsi, $B \subset X$ et $X \cap A = \varnothing$. Par suite $A\cap B = \varnothing$. Réciproquement, supposons que $A\cap B=\varnothing$ et soit $(A_1, B_1)\in \mathcal{P}(A)\times \mathcal{P}(B)$. On cherche $X\in \mathcal{P}(E)$ tel que $\Phi (X)=(A_1,B_1)$. Donc $X\cap A =A_1$ et $X\cap B = B_1$. Ainsi $A_1 \subset X$ et $B_1 \subset X$. Si l'on choisit $X$ tel que $A_1 \subset X$  et $B_1 \subset X$ alors $\Phi (X)=(A_1,B_1)$. Par exemple $\Phi (A_1\cup B_1)=(A_1,B_1)$ car $A\cap B=\varnothing$.
		\item Soit $(A_1, B_1)\in \mathcal{P}(A)\times \mathcal{P}(B)$. Comme $\Phi$ est surjective, il existe $X\in \mathcal{P}(E)$ tel que $\Phi (X)=(A_1,B_1)$. Donc $X\cap A =A_1$ et $X\cap B = B_1$. Donc $A_1 \cup B_1 =(X\cap A)\cup (X\cap B) = X\cap (A\cup B)$. Comme $\Phi$ est injective, $A\cup B=E$. Donc $X=A_1\cup B_1$. Par suite, la réciproque de $\Phi$ est définie par:
		\begin{align*}
			\Phi^{-1}: \mathcal{P}(A)\times \mathcal{P}(B) &\longrightarrow \mathcal{P}(E)\\
			(X,Y) &\longmapsto X\cup Y
		\end{align*}
	\end{enumerate}
