    \begin{enumerate}
        \item On montre sans difficulté que $S$ est réflexive, antisymétrique et transitive.
        \item La relation $S$ est une relation d'ordre partiel car, par exemple, 2 et 3 ne sont pas comparables par $S$.
        \item Ensemble: \begin{itemize}
            \item des majorants de $A:\ \{60, 120, \cdots, 60k, \cdots\}$, ce sont les multiples non nuls de 60,
            \item des majorants de $B:\ \{128, 256, \cdots, 128k, \cdots\}$, ce sont les multiples non nuls de 128,
            \item des majorants de $C:\ \{480, 960, \cdots, 480k, \cdots\}$, ce sont les multiples non nuls de 480;
        \end{itemize}
        \item Ensemble: \begin{itemize}
            \item des minorants de $A:\ \{1\}$,
            \item des minorants de $B:\ \{1, 2\}$,
            \item des minorants de $C:\ \{1, 2, 4, 8, 16\}$.
        \end{itemize}
        \item \begin{itemize}
            \item  $\text{bornesup}(A)=ppcm$ des éléments de $A$;
            \item  $\text{borneinf}(A)=pgcd$ des éléments de $A$;
        \end{itemize}
    \end{enumerate}
